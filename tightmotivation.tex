\section{The notion of tight semantics:}
\subsection{The need of tight semantics}
\label{sec:motivate-tight}

Contracts in \cdl\ are subject to \emph{when} clauses are first met or first broken.
A rent payment that occurs \emph{now} settles the duty at this exact position; any
later symbols should not re-open (nor re-justify) the same decision.  Likewise,
the first missed deadline fixes the violation point; subsequent steps are merely
after-the-fact evidence.  Before introducing the formal machinery, we motivate
the need for a \emph{tight}, prefix-based semantics that singles out these
decisive instants.

\medskip
\noindent
\textbf{What goes wrong without tightness.}
If we only tag a prefix as ``accepted'' whenever it spells a word in the target
language and keep tagging all longer extensions as ``accepted'' again, we
\emph{lose} the unique earliest acceptance point.  This leads to
(i) ambiguity about \emph{when} credit is earned,
(ii) potential ``double counting'' of compliance,
and (iii) difficulty aligning guarded/triggered clauses with the moment they
should switch on or off.  Dually, labeling every failing extension as a fresh
violation blurs \emph{when} the duty was first broken.

\medskip
\noindent
\textbf{Tiny illustration.}
Let \(\Sigma=\{a,b\}\) and \(L=\{a\}\) (``seeing \(a\) once is success'').
Reading \(a\) at the first position should \emph{decide} compliance then and
there; the longer words \(aa,ab,\dots\) must be treated as \emph{after} the
decision, not as new acceptances.  Conversely, reading \(b\) first fixes the
earliest failure; \(ba,bb,\dots\) are merely \emph{after} that failure.

\medskip
\noindent
\textbf{What tight semantics will guarantee.}
Our five-valued, prefix-oriented view will:
\begin{itemize}
  \item identify the \emph{first acceptance} index (earliest satisfaction);
  \item identify the \emph{first rejection} index (earliest violation);
  \item classify any strict extension \emph{after} these frontiers as ``post'' acceptance/rejection;
  \item mark all prefixes that are still undecided but extendable to acceptance as \emph{pre-eager}.
\end{itemize}
This yields determinacy (exactly one verdict per prefix), uniqueness of frontiers,
and monotone \emph{evolution} of verdicts along extensions—properties crucial for
correctness, fairness, and auditable timing in contracts.

\medskip
\noindent
With this motivation in place, we now introduce the language-theoretic operators
and automata constructions that realize these frontiers and, subsequently, define
the tight five-valued semantics.

\subsection{Regular Language-engineering for tight behavior}
\label{sec:lang-frontiers-tight}

\noindent
For the sake of simplicity this subsection shows how to build the motivated tight semantics on \emph{any regular languages} defined over $L\subseteq\Sigma^*$.
We introduce prefix-based operators that isolate the first decisive points of
\emph{acceptance} and \emph{rejection}, and separate them from their \emph{extension by any suffix}.
The outcome is a disjoint and complete five-way partition of $\Sigma^*$ tailored to tight semantics.
We write $u\preceq v$ when $u$ is a (possibly empty) prefix of $v$, and $u\prec v$ for a strict prefix.
For $S,X\subseteq\Sigma^*$, $SX:=\{uv\mid u\in S,\,v\in X\}$, and $\Sigma^{+}:=\Sigma^*\setminus\{\varepsilon\}$.


\begin{definition}[Prefix-closure and bad-prefix]\label{def:closure-bad}
For a language $L\subseteq\Sigma^*$,
\[
\closureclass{L} \ :=\ \bigl\{\,u\in\Sigma^* \mid \exists v\in L:\ u\preceq v\,\bigr\},
\qquad
\badclass{L} \ :=\ \Sigma^* \setminus \closureclass{L}.
\]
\end{definition}

\begin{definition}[Minimal elements]\label{def:min}
For any $S\subseteq\Sigma^*$,
\[
\minlang{S} \ :=\ \bigl\{\,u\in S \mid \nexists v\in S:\ v\prec u\,\bigr\}.
\]
In particular, $\minlang{L}$ are the \emph{first acceptance} words, and
$\minlang{\badclass{L}}$ are the \emph{first rejection} (minimal bad) words.
\end{definition}

\begin{definition}[Additional frontier definitions]\label{def:overshoots}
\[
\begin{array}{rcll}
\IAL{L}  &:=& \minlang{L}\,\Sigma^{+}                            & \text{(irrelevant acceptance)},\\[2pt]
\sbad{L} &:=& \minlang{\badclass{L}} \ \setminus\ \IAL{L}        & \text{(strict bad frontier)},\\[2pt]
\IRL{L}  &:=& \sbad{L}\,\Sigma^{+}                                & \text{(irrelevant rejection)}.
\end{array}
\]
\end{definition}

\paragraph{Why the subtraction $\minlang{\badclass{L}}\setminus \IAL{L}$?}
A word may be simultaneously a minimal bad prefix and a strict extension of a minimal
acceptance (e.g.\ $aa$ for $L=\{a\}$ over $\Sigma=\{a,b\}$).
We resolve this tie \emph{in favor of acceptance}, classifying such words as
\emph{acceptance overhead}. Removing $\IAL{L}$ from $\minlang{\badclass{L}}$
keeps the two frontiers disjoint.

\paragraph{Deconstructing regular language for tight behavior}

\begin{lemma}\label{lem:basic}
For any $L\subseteq\Sigma^*$:
\begin{enumerate}
  \item $\minlang{L}\ \subseteq\ \closureclass{L}$,\quad
        $\minlang{\badclass{L}}\ \subseteq\ \badclass{L}$,\quad
        and\quad $\minlang{L}\ \cap\ \minlang{\badclass{L}}=\emptyset$.
  \item $\closureclass{\minlang{L}}=\bigl\{\,u\mid \exists m\in\minlang{L}:\ u\preceq m\,\bigr\}
        \ =\ \bigl(\closureclass{\minlang{L}} \setminus \minlang{L}\bigr)\ \dotcup\ \minlang{L}$.
\end{enumerate}
\end{lemma}

\begin{proof}
(1) The inclusion $\minlang{L}\subseteq \closureclass{L}$ is immediate since every
$m\in\minlang{L}$ is itself a prefix of a word in $L$ (namely $m$). Likewise
$\minlang{\badclass{L}}\subseteq \badclass{L}$ holds by definition of minimality
within $\badclass{L}$. Disjointness follows because
$\closureclass{L}\cap \badclass{L}=\emptyset$.

(2) By definition of prefix-closure over the set of minimal acceptances.
The decomposition into the disjoint union with $\minlang{L}$ is immediate since
$\minlang{L}\subseteq \closureclass{\minlang{L}}$ and the difference removes
exactly the minimal elements.
\end{proof}

\paragraph*{Disjoint and complementary notation.}
We write  for the \emph{disjoint complementary union} of sets \(A\), \(B\), written $X = A \dotcup B$.
Thus \(X = A \dotcup B \dotcup C\) asserts that \(A,B,C\) are pairwise disjoint, i.e the intersection of any two different sets from $\{A,B,C\}$ is empty, and
\(X = A \cup B \cup C\).


\begin{lemma}[Two canonical splits]\label{lem:two-splits}
For any $L\subseteq\Sigma^*$,
\[
\closureclass{L} \ =\ \closureclass{\minlang{L}}\ \dot\cup\ \IAL{L},
\qquad
\badclass{L} \ =\ \sbad{L}\ \dot\cup\ \IRL{L}.
\]
\end{lemma}

\begin{proof}[Proof sketch]
For $u\in\closureclass{L}$ pick $z\in L$ with $u\preceq z$ and let $m$ be a shortest accepted prefix
of $z$; then $m\in\minlang{L}$. Either $u\preceq m$ (so $u\in\closureclass{\minlang{L}}$) or $m\prec u$
(so $u\in m\Sigma^{+}\subseteq \IAL{L}$).
For the bad side, every $u\in\badclass{L}$ has a unique shortest bad prefix $b\in\minlang{\badclass{L}}$ with $b\preceq u$;
if $b\notin \IAL{L}$ then either $u=b\in\sbad{L}$ or $u\in b\Sigma^{+}\subseteq \IRL{L}$; if $b\in\IAL{L}$ it is assigned to acceptance-overshoot by convention.
\end{proof}

\begin{lemma}[Cross disjointness]\label{lem:cross}
For any $L\subseteq\Sigma^*$,
\[
\closureclass{L}\cap \badclass{L}=\emptyset,\qquad
\IRL{L}\cap \IAL{L}=\emptyset,\qquad
\minlang{L}\cap \IRL{L}=\emptyset.
\]
\end{lemma}


\paragraph{Five semantic regions}

\begin{definition}[Five semantic regions]\label{def:five-regions}
For any $L\subseteq\Sigma^*$, define:
\[
\underbrace{\closureclass{\minlang{L}}\setminus \minlang{L}}_{\text{\emph{pre-eager-verdict}}}
\ \dot\cup\
\underbrace{\minlang{L}}_{\text{\emph{eager acceptance}}}
\ \dot\cup\
\underbrace{\sbad{L}}_{\text{\emph{eager rejection}}}
\ \dot\cup\
\underbrace{\IAL{L}}_{\text{\emph{irrelevant acceptance}}}
\ \dot\cup\
\underbrace{\IRL{L}}_{\text{\emph{irrelevant rejection}}}.
\]
\end{definition}

\begin{theorem}[Five-way partition of $\Sigma^*$]\label{thm:five-way}
For every $L\subseteq\Sigma^*$, the space of all possible words could be decomposed into:
\[
\Sigma^{*}
=\ \bigl(\closureclass{\minlang{L}}\setminus \minlang{L}\bigr)
\ \dot\cup\ \minlang{L}
\ \dot\cup\ \sbad{L}
\ \dot\cup\ \IAL{L}
\ \dot\cup\ \IRL{L}.
\]
\end{theorem}

\begin{proof}
From $\Sigma^*=\closureclass{L}\ \dot\cup\ \badclass{L}$ and Lemma~\ref{lem:two-splits},
\[
\Sigma^*=\underbrace{\closureclass{\minlang{L}}\ \dot\cup\ \IAL{L}}_{\closureclass{L}}
\ \dot\cup\
\underbrace{\sbad{L}\ \dot\cup\ \IRL{L}}_{\badclass{L}}.
\]
Now split $\closureclass{\minlang{L}}$ using Lemma~\ref{lem:basic}(2).
Cross disjointness follows from Lemma~\ref{lem:cross}.
\end{proof}

%example
\begin{example}[Five regions illustration on a simple language]
Let $\Sigma=\{a,b\}$ and $L=\{a\}$.
\begin{align*}
\closureclass{L}&=\{\varepsilon,a\},
&
\badclass{L}&=\Sigma^*\setminus\{\varepsilon,a\}=\{b,aa,ab,ba,bb,\dots\},\\
\minlang{L}&=\{a\},
&
\minlang{\badclass{L}}&=\{b,aa\}.
\end{align*}
Tie-break and overshoots:
\[
\IAL{L}=a\,\Sigma^{+}=\{aa,ab,aaa,\dots\},\qquad
\sbad{L}=\{b\},\qquad
\IRL{L}=b\,\Sigma^{+}=\{ba,bb,baa,\dots\}.
\]
Five regions:
\[
\underbrace{\{\varepsilon\}}_{\text{pre-eager-verdict}}
\ \dot\cup\
\underbrace{\{a\}}_{\text{eager acceptance}}
\ \dot\cup\
\underbrace{\{b\}}_{\text{eager rejection}}
\ \dot\cup\
\underbrace{a\Sigma^{+}}_{\text{irrelevant acceptance}}
\ \dot\cup\
\underbrace{b\Sigma^{+}}_{\text{irrelevant rejection}}
\ =\ \Sigma^*.
\]
\end{example}



\subsubsection{Tight five-valued semantics}
We now introduce a prefix-level semantics that takes values in the set
$\tightverdicts=\{\mathsf{?},\topt,\bott,\topp,\botp\}$, corresponding respectively to:
pre-eager verdict (undecided but extendable), eager acceptance (first satisfaction),
eager rejection (first violation), irrelevant acceptance (post acceptance),
and irrelevant rejection (post rejection).

\paragraph*{Shortcut notation.}
For a regular language $L\subseteq\Sigma^*$. For brevity we fix:
\[
\begin{array}{rcll}
\Pre{L} &:=& \closureclass{\minlang{L}}\setminus \minlang{L}, & \text{(pre-eager verdict)},\\[2pt]
\EAL{L} &:=& \minlang{L},                                     & \text{(eager acceptance)},\\[2pt]
\IAL{L} &:=& \minlang{L}\,\Sigma^{+},                         & \text{(irrelevant acceptance)},\\[2pt]
\ERL{L} &:=& \minlang{\badclass{L}} \ \setminus\ \IAL{L},                                      & \text{(eager rejection)},\\[2pt]
\IRL{L} &:=& \sbad{L}\,\Sigma^{+}.                            & \text{(irrelevant rejection)}.
\end{array}
\]

\begin{definition}[Five-valued prefix semantics]\label{def:five-valued-semantics}
Fix a regular language $L\subseteq\Sigma^*$. For any $u\in\Sigma^*$ define
\[
\semfive{u \vDash L}
\;:=\;
\begin{cases}
\mathsf{?}      & \text{if } u\in \Pre{L},\\[2pt]
\topt            & \text{if } u\in \EAL{L},\\[2pt]
\bott            & \text{if } u\in \ERL{L},\\[2pt]
\topp        & \text{if } u\in \IAL{L},\\[2pt]
\botp      & \text{if } u\in \IRL{L}.
\end{cases}
\]
\end{definition}

\paragraph*{Determinacy.}
By Theorem~\ref{thm:five-way}, the sets $\Pre{L}$, $\EAL{L}$, $\ERL{L}$, $\IAL{L}$, $\IRL{L}$
form a pairwise-disjoint and complete partition of $\Sigma^*$. Hence
$\semfive{u \vDash L}$ is well-defined and single-valued for every $u\in\Sigma^*$.

\begin{theorem}[Frontier and evolution properties of the tight five semantics]\label{prop:tightness-obligations-unindexed}
Fix $L\subseteq\Sigma^*$ and $u\in\Sigma^*$. Then:
\begin{itemize}
  \item[\textbf{(A)}] \textbf{Post-eager acceptance irrelevance.}
  If $\semfive{u \vDash L}=\topt$, then for every strict longer word $v$ with $u\prec v$
  we have $\semfive{v \vDash L}=\topp$.

  \item[\textbf{(B)}] \textbf{Unique eager-acceptance frontier before post-acceptance.}
  If $\semfive{u \vDash L}=\topp$, then there exists a \emph{unique} strict prefix
  $m\prec u$ such that $\semfive{m \vDash L}=\topt$.

  \item[\textbf{(C)}] \textbf{Post-rejection irrelevance.}
  If $\semfive{u \vDash L}=\bott$, then for every strict longer word $v$ with $u\prec v$
  we have $\semfive{v \vDash L}=\botp$.

  \item[\textbf{(D)}] \textbf{Unique eager-rejection frontier before post-rejection.}
  If $\semfive{u \vDash L}=\botp$, then there exists a \emph{unique} strict prefix
  $m\prec u$ such that $\semfive{m \vDash L}=\bott$.

  \item[\textbf{(E)}] \textbf{Two-way continuation (unique minimal frontiers).}
  If $\semfive{u \vDash L}=\mathsf{?}$, then there exist unique integers
  $\ell_{\topt},\ell_{\bott}\ge 1$ such that:
  \begin{itemize}
    \item there exists a unique suffix $v\in\Sigma^{\ell_{\topt}}$ with
          $\semfive{uv \vDash L}=\topt$, and for every strict prefix $v'$ of $v$
          we have $\semfive{uv' \vDash L}=\mathsf{?}$;
    \item if some extension of $u$ eventually leaves $\closureclass{L}$, then there exists
          a unique suffix $x\in\Sigma^{\ell_{\bott}}$ with $\semfive{ux \vDash L}=\bott$,
          and for every strict prefix $x'$ of $x$ we have $\semfive{ux' \vDash L}=\mathsf{?}$.
  \end{itemize}
\end{itemize}
\end{theorem}

\begin{proof}
As in the indexed setting, replace the slice $w[i,j]$ by the unindexed prefix $u$.
Key facts: $\minlang{L}$ and $\minlang{\badclass{L}}$ are prefix-free;
$\IAL{L}=\minlang{L}\,\Sigma^{+}$; $\IRL{L}=\sbad{L}\,\Sigma^{+}$; and
$\ERL{L}=\minlang{\badclass{L}}\setminus \IAL{L}$. Items \textbf{(A)}–\textbf{(D)}
follow immediately from these definitions. For \textbf{(E)}, take the unique
shortest extensions from $u$ into $\EAL{L}$ and, when reachable, into $\ERL{L}$;
uniqueness is by minimality over $\mathbb{N}$ together with prefix-freeness.
\end{proof}




\subsubsection{From Language automaton to Tight monitor construction}

This subsection explains how the deterministic language automaton obtained for a regular expression is lifted into a tight monitor. A language automaton recognises complete words of a regular expression, whereas a tight monitor must classify every prefix of every word into one of the five semantic regions. The transition structure of the automaton is therefore preserved, and only the output behaviour changes: each state receives a tight verdict according to the five-valued prefix semantics. This yields the tight monitor for regular expressions introduced in Definition~\ref{def:tsmc-re}.

We now recall the components used in this transformation and show how the DFA is turned into a Moore machine equipped with tight verdicts.


\begin{definition}[Five-region automata]\label{def:five-aut}
Let $L\subseteq\Sigma^*$ be a regular language and let
\[
  \aut(L)=(Q,\Sigma,\delta,q_0,F)
  \quad\text{with}\quad \Lang{\aut(L)}=L
\]
be a DFA for $L$. We define the following DFAs, all over the same alphabet $\Sigma$:
\begin{itemize}
  \item $\aut_{\mathrm{EA}}(L)  := (Q_{\mathrm{EA}},\Sigma,\delta_{\mathrm{EA}},q^{0}_{\mathrm{EA}},F_{\mathrm{EA}})$
        with $\Lang{\aut_{\mathrm{EA}}(L)}=\EAL{L}$.
  \item $\aut_{\mathrm{PRE}}(L) := (Q_{\mathrm{PRE}},\Sigma,\delta_{\mathrm{PRE}},q^{0}_{\mathrm{PRE}},F_{\mathrm{PRE}})$
        with $\Lang{\aut_{\mathrm{PRE}}(L)}=\Pre{L}$.
  \item $\aut_{\mathrm{IAL}}(L) := (Q_{\mathrm{IAL}},\Sigma,\delta_{\mathrm{IAL}},q^{0}_{\mathrm{IAL}},F_{\mathrm{IAL}})$
        with $\Lang{\aut_{\mathrm{IAL}}(L)}=\IAL{L}$.
  \item $\aut_{\mathrm{ER}}(L)  := (Q_{\mathrm{ER}},\Sigma,\delta_{\mathrm{ER}},q^{0}_{\mathrm{ER}},F_{\mathrm{ER}})$
        with $\Lang{\aut_{\mathrm{ER}}(L)}=\ERL{L}$.
  \item $\aut_{\mathrm{IRL}}(L) := (Q_{\mathrm{IRL}},\Sigma,\delta_{\mathrm{IRL}},q^{0}_{\mathrm{IRL}},F_{\mathrm{IRL}})$
        with $\Lang{\aut_{\mathrm{IRL}}(L)}=\IRL{L}$.
\end{itemize}
These DFAs are obtained using the constructions described in the previous subsection
(prefix-closure, complement, minimal frontiers, and right-ideal saturation).
\end{definition}

\begin{definition}[Five-region Moore machine]\label{def:five-moore}
Let $L\subseteq\Sigma^*$ be a regular language and let
\[
  \aut_{\mathrm{PRE}}(L),\,
  \aut_{\mathrm{EA}}(L),\,
  \aut_{\mathrm{ER}}(L),\,
  \aut_{\mathrm{IAL}}(L),\,
  \aut_{\mathrm{IRL}}(L)
\]
be the five DFAs from Definition~\ref{def:five-aut}, with accepting sets
$F_{\mathrm{PRE}},F_{\mathrm{EA}},F_{\mathrm{ER}},F_{\mathrm{IAL}},F_{\mathrm{IRL}}$.
The \emph{five-region Moore machine} for $L$ is the deterministic Moore machine
\[
  \mathcal{M}_{\text{5tight}}(L)
  \;:=\;
  \big(S,\ s^0,\ \Sigma,\ \tightverdicts,\ \delta,\ \lambda\big),
\]
where
\[
\begin{aligned}
S &:= Q_{\mathrm{PRE}}\times Q_{\mathrm{EA}}\times Q_{\mathrm{ER}}
      \times Q_{\mathrm{IAL}}\times Q_{\mathrm{IRL}},\\
s^0 &:= \big(q^0_{\mathrm{PRE}},\,q^0_{\mathrm{EA}},\,q^0_{\mathrm{ER}},\,
             q^0_{\mathrm{IAL}},\,q^0_{\mathrm{IRL}}\big),\\
\delta\!\big((p,e,r,a,\rho),\,\sigma\big)
&:= \big(\delta_{\mathrm{PRE}}(p,\sigma),\ \delta_{\mathrm{EA}}(e,\sigma),\
          \delta_{\mathrm{ER}}(r,\sigma),\
          \delta_{\mathrm{IAL}}(a,\sigma),\
          \delta_{\mathrm{IRL}}(\rho,\sigma)\big),\\
\lambda(p,e,r,a,\rho)
&:= \begin{cases}
      \topt & \text{if } e\in F_{\mathrm{EA}},\\
      \bott & \text{if } r\in F_{\mathrm{ER}},\\
      \topp & \text{if } a\in F_{\mathrm{IAL}},\\
      \botp & \text{if } \rho\in F_{\mathrm{IRL}},\\
      \mathsf{?} & \text{if } p\in F_{\mathrm{PRE}}.
    \end{cases}
\end{aligned}
\]
We write $\mathcal{M}_{\text{5tight}}(L)$ as a function of $L$, since $L$ uniquely determines
all components of this Moore machine.
\end{definition}



\begin{example}[Compact five-valued Moore monitor]
\label{ex:moore-tight-compact}
\small
The minimized five-output Moore machine for $L=\{a,ab,bb\}$ over $\Sigma=\{a,b\}$
produces $\tightverdicts=\{\mathsf{?},\topt,\bott,\topp,\botp\}$ according to the
unique accepting component among the five regions $\Pre{L}, \EAL{L}, \ERL{L}, \IAL{L}, \IRL{L}$.
\vspace{1ex}

\begin{figure}[h!]
\centering
\begin{tikzpicture}[
  ->, >=Stealth, node distance=20mm, semithick,
  every state/.style={rectangle,rounded corners,draw,minimum width=11mm,
    minimum height=7mm,inner sep=2pt,font=\scriptsize,align=center}
]

% --- Nodes with embedded verdicts ---
\node[initial,state,fill=gray!10]    (s0) {$s_0$\\$\mathsf{?}$};
\node[state,fill=green!18,right=26mm of s0]  (sa) {$s_a$\\$\topt$};
\node[state,fill=green!10,right=26mm of sa]  (sp) {$s_+$\\$\topp$};
\node[state,fill=gray!10,below=18mm of s0]   (sb) {$s_b$\\$\mathsf{?}$};
\node[state,fill=red!18,below=18mm of sa]    (sv) {$s_v$\\$\bott$};
\node[state,fill=red!10,below=18mm of sp]    (sm) {$s_-$\\$\botp$};

% --- Transitions ---
\path
  (s0) edge[bend left=10] node[above,pos=0.45] {$a$} (sa)
       edge[bend right=10] node[left,pos=0.45]  {$b$} (sb)
  (sb) edge[bend left=10]  node[above,pos=0.45] {$b$} (sa)
       edge[bend right=10] node[below,pos=0.45]  {$a$} (sv)
  (sa) edge node[above] {$a,b$} (sp)
  (sv) edge node[above] {$a,b$} (sm)
  (sp) edge[loop right] node {$a,b$} ()
  (sm) edge[loop right] node {$a,b$} ();

\end{tikzpicture}
\caption{Minimized five-valued Moore machine for $L=\{a,ab,bb\}$.
Each node shows its name and output verdict ($\tightverdicts$).
Green states denote satisfaction, red denote violation, gray undecided.
From $s_0$, input $a$ or $bb$ yields~$\topt$, input $ba$ yields~$\bott$,
and further inputs move to the post-frontier verdicts $\topp$ or $\botp$.}
\label{fig:moore-tight-compact}
\end{figure}
\end{example}


\begin{lemma}[Correctness of the tight-product Moore machine]\label{prop:five-moore-correct}
Let $L\subseteq\Sigma^*$ be regular and let $M_{\text{5tight}}(L)$ be as in
Definition~\ref{def:five-moore}. For every $u\in\Sigma^*$,
\[
\semfive{u \vDash L}\;=\;\lambda\!\big(\delta^{*}(s^0,u)\big),
\]
i.e., the output of $M_{\text{5tight}}(L)$ on input prefix $u$ coincides with the
five-valued semantics in Definition~\ref{def:five-valued-semantics}.
\end{lemma}

\begin{proof}
Let the five DFAs be
\[
\aut_{\mathrm{PRE}}(L),\quad
\aut_{\mathrm{EA}}(L),\quad
\aut_{\mathrm{ER}}(L),\quad
\aut_{\mathrm{IAL}}(L),\quad
\aut_{\mathrm{IRL}}(L),
\]
with recognized languages $\Pre{L}$, $\EAL{L}$, $\ERL{L}$, $\IAL{L}$, $\IRL{L}$, respectively.
By construction of $M_{\text{5tight}}(L)$, after reading $u$ the global state is
\[
\delta^{*}(s^0,u)
=\big(\delta_{\mathrm{PRE}}^{*}(q^{0}_{\mathrm{PRE}},u),\
      \delta_{\mathrm{EA}}^{*}(q^{0}_{\mathrm{EA}},u),\
      \delta_{\mathrm{ER}}^{*}(q^{0}_{\mathrm{ER}},u),\
      \delta_{\mathrm{IAL}}^{*}(q^{0}_{\mathrm{IAL}},u),\
      \delta_{\mathrm{IRL}}^{*}(q^{0}_{\mathrm{IRL}},u)\big).
\]
For each component DFA,
\[
\begin{aligned}
u\in\Pre{L}  &\iff \delta_{\mathrm{PRE}}^{*}(q^{0}_{\mathrm{PRE}},u)\in F_{\mathrm{PRE}},\\
u\in\EAL{L}  &\iff \delta_{\mathrm{EA}}^{*}(q^{0}_{\mathrm{EA}},u)\in F_{\mathrm{EA}},\\
u\in\ERL{L}  &\iff \delta_{\mathrm{ER}}^{*}(q^{0}_{\mathrm{ER}},u)\in F_{\mathrm{ER}},\\
u\in\IAL{L}  &\iff \delta_{\mathrm{IAL}}^{*}(q^{0}_{\mathrm{IAL}},u)\in F_{\mathrm{IAL}},\\
u\in\IRL{L}  &\iff \delta_{\mathrm{IRL}}^{*}(q^{0}_{\mathrm{IRL}},u)\in F_{\mathrm{IRL}}.
\end{aligned}
\]
By Theorem~\ref{thm:five-way}, the five languages form a pairwise-disjoint, complete
partition of $\Sigma^*$. Hence for each $u$ exactly one of the five memberships holds,
and $\lambda$ (by Definition~\ref{def:five-moore}) returns the unique verdict
of Definition~\ref{def:five-valued-semantics}. Therefore
$\lambda(\delta^{*}(s^0,u))=\semfive{u \vDash L}$.
\end{proof}


\begin{proposition}[Linear-size Moore machine]\label{lem:moore-linear}
Let $\aut(L)=(Q,\Sigma,\delta,q_0,F)$ be a completed DFA for $L$, and let
$M_{\text{5tight}}(L)$ be the tight five-valued Moore machine of
Definition~\ref{def:five-moore}. Then the number of reachable states of
$M_{\text{5tight}}(L)$ is linear in $|Q|$:
\[
  \bigl|S_{\mathrm{reach}}(M_{\text{5tight}}(L))\bigr|
  \;=\;
  \bigl|\mathrm{Reach}(\aut(L))\bigr|
  \;\le\; |Q|.
\]
\end{proposition}


\begin{proof}
We argue directly from the constructions of the five DFAs used in
$M_{\text{5tight}}(L)$.

\smallskip
\noindent
\emph{(Same transition graph).}
Starting from a completed DFA $\aut(L)=(Q,\Sigma,\delta,q_0,F)$:
\begin{itemize}
  \item The prefix-closure automaton $\closure{\aut(L)}=(Q,\Sigma,\delta,q_0,\Live)$
        changes only the accepting set to $\Live$ (backward reachability to $F$).
  \item Its complement for bad-prefixes $\badc{\aut(L)}=(Q,\Sigma,\delta,q_0,Q\setminus\Live)$
        flips finals, keeping $(Q,\Sigma,\delta)$.
  \item The minimal-frontier automaton $\minc{\aut(L)}=(Q,\Sigma,\delta,q_0,F_{\min})$
        selects the first accepting BFS layer; again, only finals change.
  \item The right-ideal closures for $\IAL{L}$ and $\IRL{L}$ “forward-saturate” the
        corresponding finals (every state reachable by a nonempty path becomes final),
        without altering $(Q,\Sigma,\delta)$.
  \item $\ERL{L}=\minlang{\badclass{L}}\setminus \IAL{L}$ is realized on the same
        $(Q,\Sigma,\delta,q_0)$ by taking
        $F_{\mathrm{ER}} := F_{\min(\mathrm{Bad})}\setminus F_{\mathrm{IAL}}$,
        since both operands already use $(Q,\Sigma,\delta,q_0)$.
\end{itemize}
Thus, \emph{all five DFAs share the identical transition graph $(Q,\Sigma,\delta)$}
and differ only in their accepting sets.

\smallskip
\noindent
\emph{(Diagonal runs in the product).}
Fix any $u\in\Sigma^*$. Because the underlying transition function is the same in
all five DFAs, the unique run from $q_0$ on $u$ ends in the \emph{same} state
$q=\delta^*(q_0,u)$ in each DFA. Therefore, in the product Moore machine
$M_{\text{5tight}}(L)$, the product state reached by $u$ is necessarily diagonal:
$(q,q,q,q,q)$.

\smallskip
\noindent
\emph{(Size bound).}
Let $\Reach(\aut(L)):=\{\delta^*(q_0,u)\mid u\in\Sigma^*\}$ be the set of states
reachable in $\aut(L)$. The mapping
\[
q\ \mapsto\ (q,q,q,q,q)
\]
is a bijection between $\Reach(\aut(L))$ and the set
$S_{\mathrm{reach}}(M_{\text{5tight}}(L))$ of product states reachable in the
Moore machine (by the diagonal property above). Hence
\[
     \bigl|S_{\mathrm{reach}}(\mathcal{M}_{\text{5tight}}(L))\bigr|
     \;=\;\bigl|\Reach(\aut(L))\bigr|
     \;\le\; |Q|.
\]
Therefore, the number of reachable Moore states grows at most linearly with the
size of $\aut(L)$.
\end{proof}




Additionally, each component DFA is deterministic (and can be completed), so
$\mathcal{M}_{\text{5tight}}(L)$ itself is a deterministic Moore machine over $\Sigma$ that emits
exactly one verdict from $\tightverdicts=\{\mathsf{?},\topt,\bott,\topp,\botp\}$ for every
processed prefix. The evolution of verdicts along any word follows from
Theorem~\ref{prop:tightness-obligations-unindexed}: the output leaves $\mathsf{?}$ once to
either $\topt$ or $\bott$, and then remains in the corresponding post phase $\topp$ or $\botp$.

\paragraph*{Summary}
This section developed a \emph{correct-by-construction} toolkit that turns any regular
language $L\subseteq\Sigma^*$ into the five semantic regions required by tight monitoring,
and then into a single verdicting monitor. The workflow relies only on classical
automata-theoretic constructions—completion, product, complement, and breadth-first search
on the DFA graph—used exactly as standard.

Starting from a DFA $\aut(L)$ we systematically build the prefix-closure automaton
$\closure{\aut(L)}$, which recognizes $\closureclass{L}$; its complement
$\badc{\aut(L)}$, which recognizes $\badclass{L}$; and the minimal-frontier automaton
$\minc{\aut(L)}$, which recognizes $\minlang{L}$. Two right-ideal saturations yield the
post regions $\IAL{L}$ and $\IRL{L}$, and one language difference realizes the
acceptance-first tie-break $\ERL{L} := \minlang{\badclass{L}}\setminus \IAL{L}$.

Practically, the approach is modular: each region is an ordinary DFA; scalable: construct
only the reachable part of products and minimize components; and reusable across
specifications that share the same $L$. Conceptually, it aligns the linguistic
requirements of normative systems, where norms enter into force at exact positions, with
executable monitors whose decisions are fair (no premature verdicts) and final
(no evolution to the opposite verdict). In short, standard automata technology,
assembled carefully, delivers a monitor that is \emph{correct by design}.

\subsection{Illustration through Regular expressions from \cDL}
\label{subsec:re-tight}

We demonstrate how we can define monitors for regular expressions from \cDL by first defining the language semantics of the regular expression, then via a traditional transformation a deterministic automaton. Finally we use the transformation shown in the previous subsection in order to define the 5 tight semantic monitor for the formula. 
This subsection fixes the semantics of the regular expressions that act as temporal guards in \cDL.

\subsubsection{Semantics for regular expression}
We work over the \emph{letter alphabet}
\[
\Gamma \ :=\ 2^{\Sigma_C^{(1)} \cup \Sigma_C^{(2)}},
\]
\begin{definition}[Semantics of regular expressions]\label{def:re-semantics}
The satisfaction relation for a regular expression $\re$, written $\pi \modelsre \re$,
is defined over a finite trace $\pi=\langle A_0,\ldots,A_{n-1}\rangle\in\Gamma^*$,
where each letter $A_i\in\Gamma$ is a set of actions that occurred at period $i$.  
The relation is given inductively:
\[
\begin{array}{lcl}
\pi \modelsre \mathsf{A} &\text{iff}&  |\pi| = 1\ \text{and}\ \mathsf{A}\subseteq A_0,\\[4pt]
\pi \modelsre * &\text{iff}& |\pi| = 1,\\[4pt]
\pi \modelsre \varepsilon &\text{iff}& \pi = \varepsilon,\\[4pt]
\pi \modelsre \emptyset &\text{iff}& A_0=\emptyset \nd |\pi| = 1,\\[4pt]
\pi \modelsre (\re_1 \mid \re_2) &\text{iff}& (\pi \modelsre \re_1) \;\text{or}\; (\pi \modelsre \re_2),\\[4pt]
\pi \modelsre (\re_1~;~\re_2) &\text{iff}& \exists k\leq n\leq \size{\pi} : 
  \pi[0,k] \modelsre \re_1 \;\text{and}\; \pi[k{+}1,n] \modelsre \re_2,\\[4pt]
\pi \modelsre \re^n &\text{iff}&  (\text{if } n>1 \text{ then } \pi \modelsre re;re^{n-1}) \nd (\pi \modelsre re \text{ if } n=1),\\[4pt]
\pi \modelsre \re^+ &\text{iff}& 
  \exists\, n\ge1 : \pi \modelsre \re^n.
\end{array}
\]
We write $\Lang{\re} := \{\pi\in\Gamma^* \mid \pi \modelsre \re\}$ for the language of $\re$.
\end{definition}

\noindent
\noindent
\emph{Reading the clauses.}
\begin{itemize}
  \item \textbf{Atom $\mathsf{A}$.} Matches exactly one period: $\pi=\langle A_0\rangle$ with $\mathsf{A}\subseteq A_0$.
  \item \textbf{Wildcard $*$.} Matches any single period: $\pi=\langle A_0\rangle$ for arbitrary $A_0\in\Gamma$.
  \item \textbf{Empty word $\varepsilon$.} Matches only the empty trace: $\pi=\varepsilon$.
  \item \textbf{Empty-action letter $\emptyset$.} Matches the one-period trace with no actions: $\pi=\langle \emptyset\rangle$.
  \item \textbf{Union $(re_1\mid re_2)$.} Holds iff at least one disjunct holds on the whole trace.
  \item \textbf{Sequencing $(re_1\,;\,re_2)$.} There is a split index $k$ with
        $\pi[0,k]\modelsre re_1$ and $\pi[k{+}1,n]\modelsre re_2$ (both parts finite).
  \item \textbf{Fixed power $re^n$.} Iterated sequencing of $re$ exactly $n$ times:
        $re^1\equiv re$; for $n>1$, $\pi\modelsre re^n$ iff $\pi\modelsre re\,;\,re^{n-1}$.
  \item \textbf{Kleene plus $re^+$.} Some positive iteration holds: $\exists n\ge1$ with $\pi\modelsre re^n$.
\end{itemize}
All satisfaction is defined on \emph{finite} traces, we show how to detect trigger and terminating condition on regular expression using the 5 valued semantics.





\subsubsection{Automata construction matching the denotational semantics}
\label{subsec:re-to-dfa}

We now give a concrete, standard pipeline that realizes the semantics of
Definition~\ref{def:re-semantics} \emph{exactly} by an automaton over the
alphabet $\Gamma=2^{\Sigma_C^{(1)}\cup\Sigma_C^{(2)}}$.

\paragraph*{Stage 1  Thompson-style $\varepsilon$-NFA $\aut_{\varepsilon}(re)$ (alphabet-aware).}
We build $\aut_{\varepsilon}(re)$ by structural recursion on $re$, using the
usual two distinguished states $(s_{\mathrm{in}},s_{\mathrm{out}})$ per fragment
and $\varepsilon$-transitions for wiring
(\cite{Thompson1968,HopcroftUllman2001,Sipser2012}). The only twist is how we
treat letters, since an atom $\mathsf{A}$ matches \emph{any} $\Gamma$-letter $X$
that \emph{covers} $\mathsf{A}$ (Definition~\ref{def:re-semantics}).

\begin{itemize}
  \item \textbf{Atom $\mathsf{A}\subseteq\Gamma$:} create two states
        $p\to q$ and add, for \emph{every} $X\in\Gamma$ with $\mathsf{A}\subseteq X$,
        a transition $p \xrightarrow{X} q$.
        This enforces “one period, with all actions in $\mathsf{A}$ present.”

  \item \textbf{Wildcard $*$:} create $p\xrightarrow{X} q$ for \emph{all} $X\in\Gamma$.

  \item \textbf{Empty word $\varepsilon$:} create $p \xrightarrow{\varepsilon} q$.

  \item \textbf{Empty-action letter $\emptyset$:} create a single-letter fragment
        $p \xrightarrow{\{\,\emptyset\,\}} q$ (i.e., only the $\Gamma$-letter $\emptyset$).

  \item \textbf{Union $(re_1 \mid re_2)$:} build fragments for $re_1$ and $re_2$ with
        entries/exits $(p_1,q_1)$ and $(p_2,q_2)$. Add fresh $p,q$ and wire
        $p \xrightarrow{\varepsilon} p_1$, $p \xrightarrow{\varepsilon} p_2$,
        $q_1 \xrightarrow{\varepsilon} q$, $q_2 \xrightarrow{\varepsilon} q$.

  \item \textbf{Sequencing $(re_1\,;\,re_2)$:} build $(p_1,q_1)$ and $(p_2,q_2)$, then
        add $q_1 \xrightarrow{\varepsilon} p_2$ and take $(p_1,q_2)$ as entry/exit.
        This matches the split $\pi[0,k]$ and $\pi[k{+}1,n]$ in the semantics.

  \item \textbf{Fixed power $re^n$:} unroll as $re;\cdots;re$ ($n$ times). The base
        $re^1\equiv re$.

  \item \textbf{Kleene plus $re^+$:} build $(p_1,q_1)$ for $re$, then add
        $q_1 \xrightarrow{\varepsilon} p_1$ and take $(p_1,q_1)$ as entry/exit.
        (At least one iteration is enforced by entering at $p_1$.)
\end{itemize}
Mark the global entry of the whole construction as initial, and the global exit as
accepting. The resulting NFA accepts exactly $\Lang{re}$.

\paragraph*{Stage 2  Determinization.}
Apply the standard subset construction with $\varepsilon$-closures to obtain a DFA
$\aut_D(re)=(Q,\Gamma,\delta,q_0,F)$ that recognizes the same language
(\cite{RabinScott1959,HopcroftUllman2001}).


\paragraph*{Correctness (sketch).}
By structural induction on $re$. The letter fragments implement exactly the one-step
clauses (\(\mathsf{A}\), $*$, $\emptyset$), union and sequencing are the usual
$\varepsilon$-wiring proofs, and $re^n$ (unrolled) and $re^+$ (loop back from the exit)
match the inductive clauses in Definition~\ref{def:re-semantics}. Determinization and
completion preserve language.

\medskip
\noindent
\textbf{From $\aut(re)$ to the tight layers.}
Follow the same steps from Definition~.\ref{def:five-moore}:


\begin{example}[End-to-end construction]\label{ex:end-to-end-C5}
Continuing from Example.~\ref{ex:contract-encoding}, we demonstrate the automata construction for $C_5$:

\[
\begin{aligned}
\re_{C_5} &:= *^{+}\ ;\ \{\notifterm^{(1)}\}\ ;\ *^{3},\\
\Gamma    &:= 2^{\Sigma_C^{(1)} \cup \Sigma_C^{(2)}},\\
\Sigma_C  &:= \{\PAY,\ \PAYF,\ \OCC,\ \notifrepair,\ \notifterm,\ \REPAIR\}.
\end{aligned}
\]

Figure~\ref{fig:thompson-C5} depicts the Thompson-style $\varepsilon$-NFA for $\re_{C_5}$, and Figure~\ref{fig:dfa-C5-determinized} shows its determinized and completed DFA.


\tikzset{
  ->, >=Stealth,
  node distance=18mm,
  every state/.style={minimum size=16pt,inner sep=1pt,font=\small}
}

\begin{figure}[h]
\centering
\begin{tikzpicture}
  % States
  \node[initial,state]            (s0) {$s_0$};
  \node[state,below=of s0]        (s1) {$s_1$};
  \node[state,right=of s1]        (s2) {$s_2$};
  %\node[state,right=14mm of s2]   (s1r) {$\,$}; % routing helper (invisible)
  \node[state,right=22mm of s2]   (s3) {$s_3$};
  \node[state,right=of s3]        (s4) {$s_4$};
  \node[state,above=of s4]        (s5) {$s_5$};
  \node[state,left=of s5]        (s6) {$s_6$};
  \node[state,accepting,left=of s6] (s7) {$s_7$};

  % Epsilon wiring: entry to + loop region
  \path (s0) edge[dashed] node[left] {$\varepsilon$} (s1);
  % + block: at least one Γ then possibly repeat (Thompson for re^+)
  \path (s1) edge node[above] {$*$} (s2);
  \path (s2) edge[dashed,bend left=40] node[above] {$\varepsilon$} (s1);

  % Move from the + block to the atom {notifterm^(1)}
  \path (s2) edge[dashed] node[above] {$\varepsilon$} (s3);
  % The atom itself: a single letter that contains notifterm^(1)
  \path (s3) edge node[above] {$\mathsf{T}$} (s4);

  % Then exactly 3 wildcards (three periods)
  \path (s4) edge node[right] {$*$} (s5);
  \path (s5) edge node[above] {$*$} (s6);
  \path (s6) edge node[above] {$*$} (s7);

  % Legend (optional, small)
  \node[below=14mm of s2,align=center] (leg) {%
    $*$: any letter in $2^{\Sigma}$ \quad
    $\mathsf{T}$: set $A$ with $\notifterm^{(1)}\in A$ \quad
    dashed $\,\varepsilon$: wiring
  }
  ;
\end{tikzpicture}
\caption{Thompson-style $\varepsilon$-NFA for
$\re_{C_5} = *^{+}\ ;\ \{\notifterm^{(1)}\}\ ;\ *^{3}$ over $\Gamma=2^{\Sigma}$.
From $s_0$ we enter the $*^{+}$ block ($s_1 \xrightarrow{\Gamma} s_2$ with a back
$\varepsilon$-loop to enforce “one or more” steps), then take a single
$\mathsf{T}$-labeled letter (the period that contains $\notifterm^{(1)}$),
followed by exactly three arbitrary periods (three $\Gamma$ transitions) to the
accepting state $s_7$. Determinization and completion of this NFA yield a DFA
that recognizes precisely the denotation of $\re_{C_5}$ in
Definition~\ref{def:re-semantics}.}
\label{fig:thompson-C5}
\end{figure}

\begin{figure}
\centering
\begin{tikzpicture}
  % States
  \node[initial,state]             (q0) {$q_0$};                 % start: 0 letters so far
  \node[state,below=of q0]         (qpre) {$q_{\text{pre}}$};    % consumed ≥1 letter, before the middle T
  \node[state,right=28mm of qpre]  (qT0) {$q_{T+0}$};            % just saw the middle T, need 3 more
  \node[state,right=20mm of qT0]   (qT1) {$q_{T+1}$};            % need 2 more
  \node[state,right=20mm of qT1]   (qT2) {$q_{T+2}$};            % need 1 more
  \node[state,accepting,above=22mm of qT2] (qAcc) {$q_{\mathrm{acc}}$}; % done: exactly 3 after T
  \node[state,fill=red!12,left=22mm of qAcc] (sink) {$\bot$};     % sink for any overrun / dead move

  % Transitions
  % From q0: first letter must be in *^+, so either T or \overline{T} goes to pre
  \path (q0) edge node[right] {$T,\overline{T}$} (qpre);

  % In q_pre: keep consuming *^+ by \overline{T} (stay), or take the middle T to enter the tail counter
  \path (qpre) edge[loop left] node {$\overline{T}$} ()
              edge node[above] {$T$} (qT0);

  % After the middle T: count exactly 3 more letters (any)
  \path (qT0) edge node[above] {$T,\overline{T}$} (qT1);
  \path (qT1) edge node[above] {$T,\overline{T}$} (qT2);
  \path (qT2) edge node[left]  {$T,\overline{T}$} (qAcc);

  % From accepting, any further letter overruns → sink; sink loops
  \path (qAcc) edge node[above] {$T,\overline{T}$} (sink);
  \path (sink) edge[loop left] node {$T,\overline{T}$} ();

\end{tikzpicture}
\caption{Determinized and completed DFA for \(\re_{C_5}=*^{+};\{\notifterm^{(1)}\};*^{3}\),
abstracting \(\Gamma\) by \(T:=\{A\in\Gamma\mid \notifterm^{(1)}\in A\}\) and
\(\overline{T}:=\Gamma\setminus T\).
State \(q_{\text{pre}}\) collects the initial \( *^{+}\) segment; transition on \(T\) begins
the “+3 letters” counter (\(q_{T+0}\to q_{T+1}\to q_{T+2}\to q_{\mathrm{acc}}\)).
Any overrun moves to the sink.}
\label{fig:dfa-C5-determinized}
\end{figure}
\end{example}


\subsubsection{From Language automaton to Tight monitor construction}\label{subsec:re-to-monitor}

\begin{definition}[Tight monitor construction for regular expressions]\label{def:tsmc-re}
Let $\re$ be a regular expression over the alphabet $\Gamma$ and let
$L := \Lang{\re}\subseteq\Gamma^{*}$ be its language as in Definition~\ref{def:re-semantics}.
Let
\[
  \mathcal{M}_{\text{5tight}}(L)
  \;=\;\big(S, s^0, \Gamma, \tightverdicts, \delta, \lambda\big)
\]
be the five-region Moore machine for $L$ from Definition~\ref{def:five-moore}.
The \emph{tight satisfaction monitor} for $\re$ is this Moore machine:
\[
  \tsmc_{\re}(\re) \;:=\; \mathcal{M}_{\text{5tight}}(\Lang{\re}).
\]
\end{definition}

By construction, for every prefix $u\in\Gamma^{*}$, the output of $\tsmc_{\re}(\re)$ after reading $u$
coincides with the tight five-valued semantics:
\[
  \lambda\big(\delta(s^0,u)\big) \;=\; \semfive{u \vDash \Lang{\re}}.
\]

\begin{example}[Tight monitor for $C_5$]\label{ex:moore-C5-tight}
Continuing Example~\ref{ex:end-to-end-C5}, Figure~\ref{fig:moore-C5-tight-compact} shows the compact five-valued Moore monitor obtained by applying the tight monitor construction of Definition~\ref{def:tsmc-re} to the regular expression $\re_{C_5}$.
\begin{figure}[h!]
\centering
\begin{tikzpicture}[
  ->, >=Stealth, node distance=17mm, semithick,
  every state/.style={rectangle,rounded corners,draw,
    minimum width=11mm,minimum height=7mm,
    inner sep=2pt,font=\scriptsize,align=center}
]

% --- Nodes with embedded verdicts ---
\node[initial,state,fill=gray!10]  (s0)  {$s_0$\\$\mathsf{?}$};
\node[state,fill=gray!10,below=of s0]  (g)   {$s_g$\\$\mathsf{?}$};
\node[state,fill=gray!10,right=23mm of g] (t0) {$s_0^T$\\$\mathsf{?}$};
\node[state,fill=gray!10,right=19mm of t0] (t1) {$s_1^T$\\$\mathsf{?}$};
\node[state,fill=gray!10,right=19mm of t1] (t2) {$s_2^T$\\$\mathsf{?}$};
\node[state,fill=green!18,above=16mm of t2] (acc) {$s_3^T$\\$\topt$};
\node[state,fill=green!10,left=19mm of acc] (ap) {$s_{+}$\\$\topp$};

% --- Transitions ---
\path
  (s0) edge node[right,pos=0.4] {$*$} (g)
  (g) edge[loop below] node {$\overline{T}$} ()
  (g) edge node[above,pos=0.45] {$T$} (t0)
  (t0) edge node[above,pos=0.5] {$*$} (t1)
  (t1) edge node[above,pos=0.5] {$*$} (t2)
  (t2) edge node[left,pos=0.5] {$*$} (acc)
  (acc) edge node[above,pos=0.5] {$*$} (ap)
  (ap) edge[loop above] node {$*$} ();

% --- Legend ---
\node[below=10mm of t0,align=center,font=\scriptsize] (leg){
$T := \{\,A\in\Gamma\mid \notifterm^{(1)}\in A\,\}$,\;
$\overline{T} := \Gamma\!\setminus\! T$,\;
$*$ = any letter in $\Gamma$.
};

\end{tikzpicture}
\caption{Compact five-valued Moore monitor for
$re_{C_5}=*^{+};\{\notifterm^{(1)}\};*^{3}$.
Outputs are shown inside the states ($\tightverdicts$).
All prefixes remain undecided ($\mathsf{?}$) until the first usable
termination event $T$ appears, then exactly three more steps reach
the tight satisfaction frontier~$\topt$,
and further extensions yield the post-acceptance verdict~$\topp$.
No $\bott/\botp$ states exist because no prefix is permanently rejecting.}
\label{fig:moore-C5-tight-compact}
\end{figure}




\end{example}

