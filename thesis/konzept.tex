%!TEX root = thesis.tex

\chapter{Konzept}
\label{chapter-konzept}

In diesem Kapitel wird die eigentliche Erkenntnis dieser Arbeit beschrieben. Der Aufbau dieses Kapitels hängt stark vom Thema der Arbeit ab. Die in dieser Vorlage vorgeschlagenen Kapitel sind auch nur als Vorschlag und auf keinen Fall als verbindlich zu verstehen.

Die folgenden Abschnitte dieses Kapitels enthalten Beispiele für die diversen Inhaltselemente einer Arbeit.\todo{Die Abschnitte dieses Kapitels sollten natürlich nicht so in die Arbeit übernommen werden.}

\todo[inline]{Notizen an einen selbst oder den Betreuer der Arbeit sind während der Arbeit sehr nützlich. Für die finale Version können diese Todo-Notes dann komplett aufgeblendet werden.}

\section{Quellen}

Quellen sind wichtig für gutes wissenschaftliches Arbeiten. Eine Quelle kann dabei zum Beispiel
\begin{compactitem}
  \item ein Beitrag in einer Zeitschrift \cite{MopOverview},
  \item ein Beitrag in einem Sammlungsband \cite{moore},
  \item ein Buch \cite{scala},
  \item ein Beitrag im Berichtsband einer Konferenz \cite{rltl},
  \item ein technischer Bericht \cite{bitkom},
  \item eine Dissertation \cite{Leucker02},
  \item eine Abschlussarbeit \cite{RltlConv},
  \item ein (noch) nicht veröffentlichter Artikel \cite{ptLTL} oder
  \item ein Artikel auf einer Website \cite{codecommit} sein.
\end{compactitem}

